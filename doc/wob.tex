\documentclass{article}

% Input packages & formatting
\input{template/packages}
\input{template/formatting}
\newcommand{\version}{0.2.2}

\renewcommand{\cleartooddpage}[1][]{\ignorespaces} % single side
\newcommand{\caret}{$^\wedge$}

% Other macros
\renewcommand{\^}[1]{\textsuperscript{#1}}
\renewcommand{\_}[1]{\textsubscript{#1}}

\title{\Huge Tcl Widget Objects (wob)\\\small Version \version}
\author{Alex Baker\\\small\url{https://github.com/ambaker1/wob}}
\date{\small\today}
\begin{document}
\maketitle
\begin{abstract}
\begin{center}
This package ties Tk widgets to their own TclOO objects and separate Tcl interpreters. Additionally, ``wob'' provides \textit{mainLoop}: a Tcl/Tk event loop with an interactive command line.
\end{center}
\end{abstract}

\clearpage
\section{Creating Widget Objects}
Widget objects are created from the \cmdlink{widget} class using the standard methods \textit{new} or \textit{create}. 
When a widget is created, it also creates a unique Tcl interpreter and loads in the Tk package, binding the ``close window'' event to destroy the widget object and interpreter.
Once created, \cmdlink{widget} objects act as commands with an ensemble of subcommands, or methods. 
These objects can be deleted with the method \methodlink[0]{widget}{destroy}.
\begin{syntax}
   	\command{widget} new <\$title> \\
   	widget create \$objectName <\$title>
\end{syntax}
\begin{args}
   	\$objectName & Explicit name for object. \\
   	\$title & Title of main widget window (default ``Widget'').
\end{args}
\begin{example}{Creating a widget object}
\begin{lstlisting}
set widgetObj [widget new]
\end{lstlisting}
\end{example}
\subsection{Removing Widget Objects}
The standard method \methodlink[0]{widget}{destroy} removes a widget object from the main Tcl interpreter, destroying the object, widget window, and widget interpreter. 
Closing the widget window also destroys the widget object and interpreter.

\begin{syntax}
\method{widget}{destroy}
\end{syntax}
Additionally, all widget objects can be closed with the command \cmdlink{closeAllWidgets}, or by closing the main Tcl interpreter.
\begin{syntax}
\command{closeAllWidgets}
\end{syntax}

\clearpage
\section{Building a Widget}
All interfacing with the widget is done through its corresponding interpreter. 
The main method for building a widget is \methodlink[0]{widget}{eval}, which evaluates Tcl/Tk code within the widget interpreter. 
The method behaves the same as the Tcl \textit{eval} command, but within the widget interpreter.
\begin{syntax}
   	\method{widget}{eval} \$arg1 \$arg2 ...
\end{syntax}
\begin{args}
   	\$arg1 \$arg2 ... & Arguments to be concatenated into a Tcl script to evaluate.
\end{args}
The widget's interpreter can be directly accessed with the method \methodlink[0]{widget}{interp}, for advanced introspection. 
\begin{syntax}
   	\method{widget}{interp}
\end{syntax}
\subsection{Widget Variable Access}
For convenience, variable values may be passed to the widget interpreter with the method \methodlink[0]{widget}{set}, and retrieved with \methodlink[0]{widget}{get}. 
\begin{syntax}
   	\method{widget}{set} \$varName \$value
\end{syntax}
\begin{syntax}
   	\method{widget}{get} \$varName
\end{syntax}
\begin{args}
   	\$varName & Name of variable in widget interpreter. \\
   	\$value & Value to set.
\end{args}
\begin{example}{Accessing widget variables}
\begin{lstlisting}
set widget [widget new]
$widget set x {hello world}
puts [$widget get x]
\end{lstlisting}
\tcblower
\begin{lstlisting}
hello world
\end{lstlisting}
\end{example}
\clearpage
\subsection{Widget Variable Links}
By default, variables in the widget interpreter are completely separate from the main Tcl interpreter. 
The method \methodlink[0]{widget}{vlink} creates a link between variables in the main interpreter and the widget interpreter so that their values are linked.
If \texttt{\$srcVar} does not exist, it will be initialized as blank.
\begin{syntax}
   	\method{widget}{vlink} \$srcVar \$targetVar
\end{syntax}
\begin{args}
   	\$srcVar & Variable in parent interpreter (scalar or array element). \\
   	\$targetVar & Variable in widget interpreter (scalar or array element). 
\end{args}
\begin{example}{Linking widget variables}
\begin{lstlisting}
set widget [widget new]
$widget vlink x x
set x {hello world}
puts [$widget get x]
\end{lstlisting}
\tcblower
\begin{lstlisting}
hello world
\end{lstlisting}
\end{example}
\subsection{Widget Command Aliases}
By default, the widget interpreter does not interface directly with the main Tcl interpreter. 
The method \methodlink[0]{widget}{alias} creates an alias command in the widget interpreter to access a command in the main interpreter.
\begin{syntax}
   	\method{widget}{alias} \$srcCmd \$targetCmd <\$arg1 \$arg2 ...>
\end{syntax}
\begin{args}
   	\$srcCmd & Command in widget interpreter (creates the command). \\
   	\$targetCmd & Command to link to in the main interpreter (does not create the command). \\
   	\$arg1 \$arg2 ... & Optional, prefix arguments to \texttt{\$targetCmd}.
\end{args}
\clearpage
\section{Entering the Event Loop}
In order for widget components to display and be interactive, the Tk event loop must be entered. 
Some Tk commands automatically enter the event loop, like \textit{tk\textunderscore getOpenFile}, but for the most part, the event loop must be entered with a call to \textit{vwait}, \textit{tkwait}, or \textit{update} (it is generally bad practice to use \textit{update} though, for a variety of reasons). 

The command \cmdlink{mainLoop} is provided as a method to enter the event loop for all widgets, while also taking interactive input from the command line, similar to the ``wish.exe'' Tcl/Tk program.
\begin{syntax}
   	\command{mainLoop} <\$onBlank>
\end{syntax}
\begin{args}
   	\$onBlank & What to do after user enters a blank line: ``continue'' will continue the interactive event loop, and ``break'' will exit the interactive event loop. Default ``continue''.
\end{args}

\begin{example}{Entering the event loop}
\begin{lstlisting}
puts "Main Loop:"
mainLoop
\end{lstlisting}
\tcblower
\begin{lstlisting}
Main Loop:
> |
\end{lstlisting}
\end{example}

\subsection{Exiting the Event Loop}
To exit the event loop and continue with a script, simply enter ``return'' on the command line, or use the command \cmdlink{exitMainLoop}, which can also be scheduled as an event with the Tcl \textit{after} command.
\begin{syntax}
\command{exitMainLoop} <\$option \$value ...> <\$result>
\end{syntax}
\begin{args}
\$option \$value ... & Tcl \textit{return} options. \\
\$result & Value to pass as result of \cmdlink{mainLoop}.
\end{args}

\clearpage
\section{Basic Applications}
The example below demonstrates how the wob package can be used to create and manipulate Tk widgets.
\begin{example}{Filename dialog}
\begin{lstlisting}
set widget [widget new]
set filename [$widget eval tk_getOpenFile]
$widget destroy
puts $filename
\end{lstlisting}
\end{example}

\begin{example}{Option selection}
\begin{lstlisting}
set widget [widget new]
$widget eval {
	label .label -text "Choose analysis type:"
	tk_optionMenu .options AnalysisType "" Pushover Dynamic
	pack .label -side top -fill x
	pack .options -side bottom -fill x
	vwait AnalysisType
}
puts [$widget get AnalysisType]
$widget destroy
\end{lstlisting}
\end{example}

\begin{example}{Access clipboard}
\begin{lstlisting}
set widget [widget new]
$widget set text "hello world"
$widget eval {
	clipboard clear
	clipboard append $text
}
mainLoop
# now the text "hello world" can be pasted into another application.
\end{lstlisting}
\end{example}

\end{document}