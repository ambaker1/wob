\documentclass{article}

% Input packages & formatting
% Packages

% Math packages
\usepackage{amsmath} % Extended math functions
\usepackage{amssymb} % Extended math symbols (loads in amsfonts)
\usepackage{bm} % Bold math symbols
\usepackage{mathtools}

% Figure packages
\usepackage{caption} % Caption formatting for university standard
\usepackage{graphicx} % includegraphics command
\usepackage{subcaption} % Subfigures
\usepackage[section]{placeins} % Place floats in section
\usepackage{wrapfig}

% Table packages
\usepackage{booktabs} % Better tables
\usepackage{bigstrut} % Merged table cells
\usepackage{longtable} % Tables which overflow into next page
\usepackage{array}
\usepackage{colortbl} % Color table cells
\usepackage{makecell}
\usepackage{multirow}

% Fonts
\usepackage{lmodern} % Use latin modern rather than computer modern. Better for font encoding.
\usepackage[T1]{fontenc} % Allow text to be searchable in output

% Other packages
\usepackage{appendix} % Appendix environment
\usepackage{nextpage} % Cleartooddpage command
%\usepackage[square,comma,sort,numbers]{natbib} % Reference formatting
\usepackage{setspace} % Line spacing
\usepackage{listings} % Display code with syntax highlighting
\usepackage{upquote} % Vertical quotes in verbatim
\usepackage{xcolor} % Colors
\usepackage{titlesec} % Header spacing
\usepackage{xparse} % for tcolorbox
\usepackage[listings]{tcolorbox} % Colored boxes for highlighting syntax
\tcbuselibrary{breakable}
\tcbuselibrary{skins}
\usepackage{enumitem} % better enumerate/itemize options
\usepackage{fancyhdr}
\usepackage{multicol}
\usepackage{ifthen}
\usepackage{xstring}

% Table of contents
\usepackage{imakeidx} % Index page
\usepackage{tocloft} % Control of table of contents
\usepackage[nottoc]{tocbibind} % Adds bibliography, table of tables, table of figures, to table of contents
\usepackage[bookmarks,linktocpage,hidelinks]{hyperref} % Hyperlinks for sections, figures, etc.

% Formatting
% Page format
\setlength{\oddsidemargin}{0.00in}  % Left side margin for odd numbered pages
\setlength{\evensidemargin}{0.00in} % Right side margin for even numbered pages
\setlength{\topmargin}{0.00in}      % Top margin
\setlength{\headheight}{0.20in}     % Header height
\setlength{\headsep}{0.20in}        % Separation between header and main text
\setlength{\topskip}{0.00in}        % Top skip
\setlength{\textwidth}{6.50in}      % Width of the text
\setlength{\textheight}{8.50in}     % Height of the text
\setlength{\footskip}{0.50in}       % Foot skip
\setlength{\parindent}{0.00in}      % First line indentation
\setlength{\parskip}{6pt}        % Space between two paragraphs

% Captions (figures, tables, etc.)
\setlength{\floatsep}{\parskip}          % Space left between floats.
\setlength{\textfloatsep}{\floatsep}   % Space between last top float
% or first bottom float and the text
\setlength{\intextsep}{\floatsep}      % Space left on top and bottom
% of an in-text float
\setlength{\abovecaptionskip}{0.1in plus 0.25in}  % Space above caption
\setlength{\belowcaptionskip}{0.1in plus 0.25in}  % Space below caption
\setlength{\captionmargin}{0.50in}     % Left/Right margin for caption
\setlength{\abovedisplayskip}{0.00in plus 0.25in} % Space before Math stuff
\setlength{\belowdisplayskip}{0.00in plus 0.25in} % Space after Math stuff
\setlength{\arraycolsep}{0.10in}       % Gap between columns of an array
\setlength{\jot}{0.10in}                % Gap between multiline equations
\setlength{\itemsep}{0.10in}           % Space between successive items

% Counters (no section numbering)
\setcounter{tocdepth}{3}
\setcounter{secnumdepth}{0}

% Spacing
\setstretch{1.5}

\titlespacing*{\section}{0cm}{6pt}{6pt}[0cm]
\titlespacing*{\subsection}{0cm}{6pt}{6pt}[0cm]
\titlespacing*{\subsubsection}{0cm}{6pt}{6pt}[0cm]

\titleformat{\section}
{\sffamily\huge}{}{0pt}{\titlerule\vspace{-0.2cm}}
\titleformat{\subsection}
{\sffamily\itshape\Large}{}{0pt}{}

% Macro for syntax
\newtcolorbox{syntax}{
    size=small,
    sharp corners,
    colframe=black,
    colback=yellow,
    fontupper=\bfseries\ttfamily
}

% Macro for argument table
\newenvironment{args}{
    \begin{tabular}{>{\bfseries\ttfamily}p{0.25\linewidth} p{0.69\linewidth}}
    }{
    \end{tabular}\par
    \vspace{0.5\baselineskip}
}

% Note: Requires packages "listing", "xcolor", and "textcomp"
\lstdefinelanguage{verbatim}{
    basicstyle=\ttfamily\small,
    xleftmargin=9pt,
    xrightmargin=9pt,
    columns=fullflexible,
    keepspaces=true,
    breaklines=true
}

% Example code
\AtBeginDocument{
\newtcolorbox[blend into=listings]{example}[2][]{
    colback=blue!3!white,
    colframe=black,
    colbacktitle=blue!15!white,
    coltitle=black,
    sharp corners,
    enhanced,
    breakable,
    size=small,
    before upper={
        \setstretch{1.0}\lstset{language=verbatim}\vspace{3pt}\textsf{\textit{Code:}}
    },
    subtitle style={
        colback=blue!20!white,
        fonttitle=\sffamily
    },
    before lower={
        \setstretch{1.0}\lstset{language=verbatim}\vspace{3pt}\textsf{\textit{Output:}}
    },
    fonttitle=\sffamily,
    title={#2},
    #1
}
}

% Links to sub and subsub commands - optional boolean argument, default true. if false, only displays subcmd.

% Commands (and command ensembles)
\newcommand{\command}[1]{\protect\hypertarget{#1}{#1}\index{#1}}
\newcommand{\subcommand}[2]{\protect\hypertarget{#1 #2}{#1 #2}\index{#1!#2}}
\newcommand{\cmdlink}[1]{\protect\hyperlink{#1}{\textit{#1}}}
\newcommand{\subcmdlink}[3][1]{\protect\hyperlink{#2 #3}{\ifnum#1=1\relax\textit{#2 #3}\else\textit{#3}\fi}}

% Methods (first arg is class)
\newcommand{\method}[2]{\protect\hypertarget{$#1Obj #2}{\$#1Obj #2}\index{#1 methods!#2}}
\newcommand{\methodlink}[3][1]{\protect\hyperlink{$#2Obj #3}{\ifnum#1=1\relax\textit{\$#2Obj #3}\else\textit{#3}\fi}}

% Macros for figure/table names
\newcommand{\fig}{\figurename\ }
\newcommand{\figs}{\figurename s }
\newcommand{\tbl}{\tablename\ }
\newcommand{\tbls}{\tablename s }
\newcommand{\eq}{Eq. }
\newcommand{\eqs}{Eqs. }
\renewcommand{\lstlistingname}{Example}% Listing -> Example
\renewcommand{\lstlistlistingname}{List of \lstlistingname s}% List of Listings -> List of Examples
\newcommand{\ex}{Example }
\newcommand{\exs}{Examples }
\newcommand{\var}[1]{\texttt{\textbf{\$#1}}}

% Header/footer
\renewcommand{\headrulewidth}{0pt}

% Changes to hyperlinks (URLs)
\renewcommand\UrlFont{\color{blue}\rmfamily}

% New column type 
% https://tex.stackexchange.com/questions/75717/how-can-i-mix-itemize-and-tabular-environments
\newcolumntype{L}{>{\labelitemi~~}l<{}}
\renewcommand{\cleartooddpage}[1][]{\ignorespaces} % single side
\newcommand{\caret}{$^\wedge$}

% Other macros
\renewcommand{\^}[1]{\textsuperscript{#1}}
\renewcommand{\_}[1]{\textsubscript{#1}}

\title{\Huge Tcl Widget Objects\\\small Version 0.1.0}
\author{Alex Baker\\\small\hyperlink{https://github.com/ambaker1/wob}{https://github.com/ambaker1/wob}}
\date{\small\today}
\begin{document}
\maketitle
\begin{abstract}
Due a conflict between the OpenSees and Tcl \textit{load} commands, Tcl binary packages cannot be loaded in after \textit{model} is called. Most notably, this restricts the use of Tk widgets. 
However, the Tcl \textit{interp} command can be used to create a fresh Tcl interpreter within the main OpenSees interpreter, and widgets can be built there instead. 
The ``wob'' package formalizes this, allowing for creation of Tk widget objects, each with their own Tcl interpreter.

Note that this package is simply a framework to build widgets with. 
Knowledge of Tk and event-driven programming is critical to build widgets.
Also, although this package was developed specifically for OpenSees, it is still applicable to other Tcl applications.
\end{abstract}

\clearpage
\section{Creating Widget Objects}
Widget objects are created from the \cmdlink{widget} class using the standard methods \textit{new} or \textit{create}. 
When a widget is created, it also creates a unique Tcl interpreter and loads in the Tk package, binding the ``close window'' event to destroy the widget object and interpreter.
Once created, \cmdlink{widget} objects act as commands with an ensemble of subcommands, or methods. 
These objects can be deleted with the method \methodlink[0]{widget}{destroy}.
\begin{syntax}
   	\command{widget} new <\$title> \\
   	widget create \$objectName <\$title>
\end{syntax}
\begin{args}
   	\$objectName & Explicit name for object. \\
   	\$title & Title of main widget window (default ``Widget'').
\end{args}
\begin{example}{Creating a widget object}
\begin{lstlisting}
set widgetObj [widget new]
\end{lstlisting}
\end{example}
\subsection{Removing Widget Objects}
The standard method \methodlink[0]{widget}{destroy} removes a widget object from the main OpenSees interpreter, destroying the object, widget window, and widget interpreter.
This is also called if the window is closed or an ``exit'' statement is evaluated in the widget interpreter.
\begin{syntax}
   	\method{widget}{destroy}
\end{syntax}
\subsection{The Widget Interpreter}
All interfacing with the widget is done through its corresponding interpreter. 
The widget's interpreter command can be accessed with the method \methodlink[0]{widget}{interp}, for advanced introspection. 
\begin{syntax}
   	\method{widget}{interp}
\end{syntax}
\clearpage
\section{Building a Widget}
The main method for building a widget is \methodlink[0]{widget}{eval}, which evaluates Tcl/Tk code within the widget interpreter. 
The method behaves the same as the Tcl \textit{eval} command, but within the widget interpreter.
\begin{syntax}
   	\method{widget}{eval} \$arg1 \$arg2 ...
\end{syntax}
\begin{args}
   	\$arg1 \$arg2 ... & Arguments to be concatenated into a Tcl script to evaluate.
\end{args}
\subsection{Widget Variable Access}
For convenience, variable values may be passed to the widget interpreter with the method \methodlink[0]{widget}{set}, and retrieved with \methodlink[0]{widget}{get}. 
\begin{syntax}
   	\method{widget}{set} \$varName \$value
\end{syntax}
\begin{syntax}
   	\method{widget}{get} \$varName
\end{syntax}
\begin{args}
   	\$varName & Name of variable in widget interpreter. \\
   	\$value & Value to set.
\end{args}
\subsection{Widget Command Aliases}
By default, the widget interpreter does not interface directly with the main OpenSees interpreter. The method \methodlink[0]{widget}{alias} creates an alias command in the widget interpreter to access a command in the main interpreter.
This is identical to the Tcl \textit{interp} method.
\begin{syntax}
   	\method{widget}{alias} \$srcCmd \$targetCmd <\$arg1 \$arg2 ...>
\end{syntax}
\begin{args}
   	\$srcCmd & Command in widget interpreter (creates the command). \\
   	\$targetCmd & Command to link to in the main interpreter (does not create the command). \\
   	\$arg1 \$arg2 ... & Optional, prefix arguments to \texttt{\$targetCmd}.
\end{args}
\clearpage
\section{Entering the Event Loop}
In order for widget components to display and be interactive, the Tk event loop must be entered. 
Some Tk commands automatically enter the event loop, like \textit{tk\textunderscore getOpenFile}, but for the most part, the event loop must be entered with a call to \textit{vwait}, \textit{tkwait}, or \textit{update} (it is generally bad practice to use \textit{update} though, for a variety of reasons). 

The command \cmdlink{mainLoop} is provided as a method to enter the event loop for all widgets, while also taking interactive input from the command line, similar to the ``wish.exe'' Tcl/Tk program.
To exit the event loop and continue with a script, simply enter ``return'' on the command line.
\begin{syntax}
   	\command{mainLoop} <\$onBlank>
\end{syntax}
\begin{args}
   	\$onBlank & What to do after user enters a blank line: ``continue'' will continue the interactive event loop, and ``break'' will exit the interactive event loop. Default ``continue''.
\end{args}

\clearpage
\section{Basic Applications}
The example below demonstrates how the wob package can be used to create and manipulate Tk widgets.
\begin{example}{Filename dialog}
\begin{lstlisting}
set widget [widget new]
set filename [$widget eval tk_getOpenFile]
$widget destroy
\end{lstlisting}
\end{example}

\begin{example}{Option selection}
\begin{lstlisting}
set widget [widget new]
$widget eval {
	label .label -text "Choose analysis type:"
	tk_optionMenu .options AnalysisType "" Pushover Dynamic
	pack .label -side top -fill x
	pack .options -side bottom -fill x
	vwait AnalysisType
}
puts [$widget get AnalysisType]
$widget destroy
\end{lstlisting}
\end{example}

\begin{example}{Access clipboard}
\begin{lstlisting}
set widget [widget new]
$widget set text "hello world"
$widget eval {
	clipboard clear
	clipboard append $text
}
$widget destroy
\end{lstlisting}
\end{example}

\end{document}